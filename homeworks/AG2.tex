\section{Exercise 1}

\begin{enumerate}[a)]
 %\stepcounter{enumi}
 \item Many of you used that $\forall m\in M f^{-1}(f(m))=m$. That is not true unless f is bijective (but it was not specified here).
 
 For instance if you take $f:\RR\to\RR,x\mapsto 0$, then $\forall x\in\RR,\, f^{-1}(f(\{m\}))=\RR\neq \{m\}$.
 
 \item Here, most of you missed the point. The goal was not to prove that $xR_f y:\iff(x)=f(y)$ is an equivalence relation, but that the equivalence classes of $R_f$ are exactly thoses of $\mathcal{P}_f$.
 
 Since you proved that, it is straightforward that $R_f$ is an equivalence relation as $\mathcal{P}_f$ is a partition of $M$.
\end{enumerate}

\section{Exercise 2}

 Most of you proved linearity of each of the functions ``by hand''. It was not necessary here, since if you prove that a function has an underlying matrix, it is straightforward that the function is linear.

\section{Exercise 3}

\begin{enumerate}[a)]
 \item Many of you did not correctly used the hypothesis in the implication round. In particular, it is completly useless to write $\Phi$ as $\Phi(X)=\begin{pmatrix}a & b\\c & d\end{pmatrix}X$ and in almost all cases it induces a false reasoning.

 Even if the statements to prove are easy and obvious (as it was here), you have to stick to the hypothesis and to find the good way of proving it, since this was what really matters in this exercise.
 
 \item This exercise was harder compared to the others and therefore was not very successful.
 
 A common error was to set conditions on $s$ and $t$, which is not what was intended. The point was to prove that the image of a parallelogram under $\Phi$ maps to a parallelogram, a segment or a point.
 Then, the conditions applies to $\Phi$, not the initial parallelogram.
 
 \textit{Ich rate Ihnen, die Korrektur der Übung 2 des zweiten Arbeitsblattes zu lesen und diese Frage dann zu überarbeiten.}
\end{enumerate}

\section{Exercise 4}

Most of you apparently knew the answer, but most failed to prove it correctly. Here the thing was to use the geometric knowledge of highschool (\textit{Gymnasium}) to ``illustrate'' it with a figure and then compute from it (as you did not, and won't in this Vorlesung, define a proper notion of area).
