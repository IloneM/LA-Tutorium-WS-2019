\section{Exercise 1}

\begin{enumerate}[a)]
 \stepcounter{enumi}
 \item As $(-1)^2+2^2=5$, we get from the definition of $R$ that every point $(x,y)\in\RR^2$ s.t (such that) $(x,y)R(-1,2)$ satisfies the equation $x^2+y^2=5$.
 
 We recognize here the cartesian equation of a circle (Kreise) of radius $\sqrt{5}$ and center $(0,0)$, which is then the equivalence class of $(-1,2)$.
 
 \item As stated in question above, the equivalence class of a point $(a,b)\in\RR^2$ is the circle of radius $\sqrt{a^2+b^2}$ and center $(0,0)$.
 
 A representation system has then to take exactly one point in every of those circles. A smart and simple choice could be to fix one of the coordinates to 0, let say the second one. The representation system is then the ray $\RR_{\geq 0}\times \{0\}\subset \RR^2$, the represent of the equivalence class of $(a,b)\in\RR^2$ being $(\sqrt{a^2+b^2}, 0)$.
 
 \textit{See exercise 4 for a more comprehensive description of what happens}
\end{enumerate}

\section{Exercise 2}

You have to work with the equivalence relation $\sim$ given by $x\sim y:\iff f(x)=f(y)$.

\section{Exercise 3}

Let denote $\forall x,y\in\RR^2$ : $[x,y]:=\{tx+(1-t)y|t\in\RR\}$, i.e ``the line between x and y''.
We should remark that when $x=y$, $[x,y]$ is nothing but the point $\{x\}=\{y\}$.\\

Let $\Phi:\RR^2\to\RR^2$ be a linear fonction, and $a, b\in\RR^2$ s.t $a\neq b$.

$\forall t\in [0,1],\, \Phi(ta+(1-t)b)=\Phi(ta)+\Phi((1-t)b)=t\Phi(a)+(1-t)\Phi(b)$ by linearity of $\Phi$, which means:
$$
 \Phi([a,b])=[\Phi(a),\Phi(b)]
$$

As $a\neq b$ by hypothesis, $[a,b]$ is here a true line (i.e not reduced to a point).

For $\Phi([a,b])=[\Phi(a),\Phi(b)]$:

\begin{itemize}
 \item if $\Phi(a)\neq\Phi(b)$, then $[\Phi(a),\Phi(b)]$ is a true line also.
 \item if $\Phi(a)=\Phi(b)$, then $[\Phi(a), \Phi(b)]$ is a reduced to the point $\{\Phi(a)\}=\{\Phi(b)\}$.
\end{itemize}

\section{Exercise 4}

\begin{enumerate}[a)]
 \item Let be $x=(x_1,x_2)\in\RR^2$. Then $r$ is given by $r=\sqrt{x_1^2+x_2^2}$.
 
 We then define $\tilde{x}=(\tilde{x}_1,\tilde{x}_2):=x/r$ and remark that $\tilde{x}_1^2+\tilde{x}_2^2=1$ which means that $\exists \theta\in [0,2\pi)$ s.t $(\tilde{x}_1,\tilde{x}_2)=(\cos(\theta),\sin(\theta))$.
 
 \textit{Ich weiss nicht ob Ihr habt schon das beweisst. Wenn nicht, k\"onnen wir das zusammen beweissen}
\end{enumerate}
